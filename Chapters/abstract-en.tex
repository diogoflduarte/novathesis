%!TEX root = ../template.tex
%%%%%%%%%%%%%%%%%%%%%%%%%%%%%%%%%%%%%%%%%%%%%%%%%%%%%%%%%%%%%%%%%%%%
%% abstract-en.tex
%% NOVA thesis document file
%%
%% Abstract in English([^%]*)
%%%%%%%%%%%%%%%%%%%%%%%%%%%%%%%%%%%%%%%%%%%%%%%%%%%%%%%%%%%%%%%%%%%%

\typeout{NT FILE abstract-en.tex}%


Fine and coordinated movements require either fast feedback control signals or robust, yet flexible feed-forward control. The cerebellum, a phylogenetically old brain structure, is thought to be the locus for learning and tuning internal models for movement.
Its architecture has been hypothesized as an efficient arrangement for learning, and its necessity for learning and coordination has been shown by means of lesion or transient circuit perturbations.
\par However the encoding of body movements and how those signals are used for coordination remain unanswered, in particular for complex behaviors. Purkinje cells, the hypothesized computational unit of the cerebellum, are known to be modulated by locomotion events, But how are these signals used and transformed at the population level?
\par In this thesis, I record from multiple cerebellar neurons across cell types during mouse locomotion. From this data I show the diversity of neural responses to locomotion across and within cell types. Additionally, I compare behavioral readouts from populations of cell types and how they vary on a trial by trial basis.
\par This work highlights how differently locomotion signals are encoded across the cerebellar circuit, something that was yet to be characterized across populations of neurons. Moreover, it establishes the baseline for future investigation of how the cerebellum computations might act during motor learning.

\keywords{
  Cerebellum \and
  Locomotion \and
  Purkinje cells \and
  Mossy fibers \and
  Coordination
}
